\renewcommand{\baselinestretch}{1.5} %設定行距
\pagenumbering{roman} %設定頁數為羅馬數字
\clearpage  %設定頁數開始編譯
\sectionef
\addcontentsline{toc}{chapter}{摘~~~要} %將摘要加入目錄
\begin{center}
\LARGE\textbf{摘~~要}\\
\end{center}
\begin{flushleft}
\fontsize{14pt}{20pt}\sectionef\hspace{12pt}\quad 專案二的目標是進行協同設計,將雙輪車應用於機器人足球比賽中。團隊組成包含4名成員,使用CAD軟體進行場景和雙輪車零組件的設計。專案的主要焦點是在足球場景中控制雙輪車進行比賽,並確保每組有4名輪車球員參與比賽。\\[12pt]

\fontsize{14pt}{20pt}\sectionef\hspace{12pt}\quad 此外,團隊還將設計單機記分板,以便在比賽中追蹤並顯示比分。記分板將使用LED顯示器來顯示分數,以便參與者和觀眾能夠隨時瞭解比賽的進展。\\[12pt]

\end{flushleft}
\begin{center}
\fontsize{14pt}{20pt}\selectfont 透過專案二的協同設計,團隊將有機會實踐協同工作並應用雙輪車於足球比賽。這將提供團隊成員們在實際場景中運用他們的技術和創造力的機會,同時為未來的機器人足球比賽項目提供基礎和啟示。
\end{center}
\newpage
%=--------------------Abstract----------------------=%
\renewcommand{\baselinestretch}{1.5} %設定行距
\addcontentsline{toc}{chapter}{Abstract} %將摘要加入目錄
\begin{center}
\LARGE\textbf\sectionef{Abstract}\\
\begin{flushleft}
\fontsize{14pt}{16pt}\sectionef\hspace{12pt}\quad Due to the four major development trends of multidimensional arrays  computing, automatic differentiation, open source development environment, and multi-core GPUs computing hardware. The rapid development of the AI field has been promoted. In view of this development, the physical mechatronic systems can gain machine learning efficiency through their simulated virtual system training process. And afterwards to apply the trained model into real mechatronic systems.\\[12pt]

\fontsize{14pt}{16pt}\sectionef\hspace{12pt}\quad This project is to use the physical air hockey to play machine, introduce it into the CoppeliaSim simulation environment and give the corresponding settings, simplify its electromechanical system and use Open AI Gym for training, find an algorithm suitable for this system, and then perform it in the CoppeliaSim simulation environment Feasibility of testing algorithm in practical application. And try to stream CoppeliaSim images to web pages for users to watch or manipulate by setting up a server.\\
\end{flushleft}
\begin{center}
\fontsize{14pt}{16pt}\selectfont\sectionef Keyword:  nerual network、reinforcement learning、 CoppeliaSim、OpenAI Gym
\end{center}