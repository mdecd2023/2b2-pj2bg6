\chapter{W10}
\section{第一題}
What is zmqRemoteAPI, and how does it relate to CoppeliaSim?

答

 zmqRemoteAPI 是一種遠端 API(應用程式編程介面),允許使用外部編寫的程式(例如 Python、C++ 或 Matlab)連接到機器人模擬軟體 CoppeliaSim。zmqRemoteAPI 使用 ZeroMQ 網路庫進行通訊,讓使用者能夠即時互動仿真環境。


\section{第二題}
How do you establish a connection between a Python script and CoppeliaSim using zmqRemoteAPI?

答

python先下載zmq子模組,利用port:23000連接


\section{第三題}
What are some common use cases for zmqRemoteAPI in CoppeliaSim?

答

在 CoppeliaSim 中使用 zmqRemoteAPI 的一些常見用例包括:

機器人模擬的即時控制
資料收集和分析
仿真資料的可視化
控制演算法的測試和驗證


\section{第四題}
What are the advantages and disadvantages of using zmqRemoteAPI compared to other methods of communication between Python and CoppeliaSim?

答

快速高效的通訊
支援多種程式語言
即時互動仿真環境
允許遠端存取仿真
缺點可能包括更陡峭的學習曲線和需要額外的程式庫。

相較於其他通訊方法,使用 zmqRemoteAPI 需要較多的程式設計經驗和了解 ZeroMQ 網路庫的概念。另外,使用 zmqRemoteAPI 需要安裝和配置 ZeroMQ 網路庫,這可能需要額外的時間和資源。

\section{第五題}
Can you give an example of a project or task that you could complete using zmqRemoteAPI in CoppeliaSim?

答

使用 zmqRemoteAPI,可以在 CoppeliaSim 中完成許多專案或任務。以下是一個使用 zmqRemoteAPI 在 CoppeliaSim 中進行視覺感知的範例:

假設有一個機器人模型,可以透過 CoppeliaSim 遠端 API 伺服器控制。此外,模型上有一個攝影機,可以捕捉模擬環境中的影像。目標是讓機器人模型能夠偵測影像中的物體並移動到它們的位置。

以下是一個使用 Python 和 zmqRemoteAPI 實現的範例:

1.在 CoppeliaSim 中建立一個機器人模型和一個攝影機。

2.在 Python 中使用 zmqRemoteAPI 建立連接,並使用 "simxGetObjectHandle" 函數獲取模型和攝影機的句柄。

3.使用 "simxGetVisionSensorImage" 函數獲取攝影機的畫面。

4.使用 OpenCV 或其他圖像處理程式庫分析攝影機畫面,偵測物體的位置。

5.使用 "simxSetJointTargetPosition" 函數控制機器人模型的移動,使其移動到偵測到的物體的位置。




\section{小組工作分配}
41023219: 

在旁邊研究Brython程式環境

41023228: 

討論與設計程式,製作亂數。
\section{2b網站順序亂數}
\begin{lstlisting}[language=Python, frame=single, numbers=left, captionpos=b, basicstyle=\ttfamily\small, showstringspaces=false, breaklines=true, tabsize=4, xleftmargin=15pt]
from browser import html, document
import random
bcd_tem = "https://mdecd2023.github.io/2b2-pj2bg"
bgithub = "https://github.com/mdecd2023/2b2-pj2bg"
brython_div = document["brython_div1"]
#  亂數範圍從1到16
grp = []
for i in range(1, 17):
    grp.append(i)
random.shuffle(grp)
for i in grp:
    url = bcd_tem + str(i)
    github = bgithub + str(i)
    brython_div <= html.A("pj2bg"+str(i), href=url)
    brython_div <= " ("
    brython_div <= html.A("repo", href=github)
    brython_div <= ")"
    brython_div <= html.BR()
\end{lstlisting}

41023221: 

跟組長一起討論以及改善關於Brython程式的問題

41023222: 

試著幫組長找尋亂數採樣無法正常執行的原因,試著採用其他方法。

