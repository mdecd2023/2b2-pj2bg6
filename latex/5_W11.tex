\chapter{W11}


\section{41023220}

負責部分:跟組長一起討論作業的開發內容並實測,並教導其他不熟悉的組員\\過程:上次只有把單純的足球場做出來,並沒有計分的規則,在做出記分板以及在寫記分板的程式的時候遇到了不少的問題,但我們都一一克服了\\自評:這次的分組讓我學會了如何解決上傳合併的衝突,然後在修改計分系統的時候遇到了一些問題,很感謝組長的協助。\\自評分數:62分


\section{41023226}

負責部分: 將他們fork後pull request 發生問題時(例:無法auto-merge),負責修改錯誤文件並使其能成功合併、足球場繪製、單機計分板程式設計、做出 PDF 報告\\遇到的難題和心得

1.將計分板程式弄上去後發現球進不會感應,但機器人會,原本以為我的程式出現錯誤,但沒想到是球與機器人的 Collidable(可碰撞) Measurable(可测量) Detectable(可檢測) 設定有誤,應將球的勾選而不是機器人

2.打算將進球後進球方與被進球方放回原位,並將球放在離被進球方較近的位置,結果對 'sim.getObjectPosition' 函數的未知鬧出許多笑話,像是 -1 不知道要放在座標前面還是後面XD

3. 原本打算一開始球以亂數形式產生(避免在正中間而造成雙方僵持不下),但我的記分板中的程式由於第二點以至於一開始一定要有一顆球,而不能我想到的(放進zmq程式中亂數產球)而不了了之,希望之後我能想出個好方法。

4.控制手感是真的差\\自評:遇到需多困境與難題,所幸最後都能有所收穫。\\自評分數:85分

\section{41023233}

負責部分:與組長一起設計球場,還有記分板與程式的設置\\自評:一開始有遇到問題,但是藉由組長組員一起討論還有問別人後,問題有解決\\自評分數:60分


\section{41023253}

負責部分:球門繪製\\心得:在這次的協同作業裡我利用onshape繪製球門場景,從這次作業裡我更加認識了zmqRemoteApi實作的重要性,雖然在計分板場景中因還不熟悉細部操作於是先在旁研究組長是如何做出和除錯的。\\自評分數:60分